%%fakesection Theorems
\theoremstyle{definition}
\mdfdefinestyle{mdbluebox}{%
	roundcorner = 10pt,
	linewidth=1pt,
	skipabove=12pt,
	innerbottommargin=9pt,
	skipbelow=2pt,
	nobreak=true,
	linecolor=black,
	backgroundcolor=TealBlue!5,
}
\declaretheoremstyle[
	headfont=\sffamily\bfseries\color{MidnightBlue},
	mdframed={style=mdbluebox},
	headpunct={\\[3pt]},
	postheadspace={0pt}
]{thmbluebox}

\mdfdefinestyle{mdredbox}{%
	linewidth=0.5pt,
	skipabove=12pt,
	frametitleaboveskip=5pt,
	frametitlebelowskip=0pt,
	skipbelow=2pt,
	frametitlefont=\bfseries,
	innertopmargin=4pt,
	innerbottommargin=8pt,
	nobreak=true,
	linecolor=RawSienna,
	backgroundcolor=Salmon!5,
}
\declaretheoremstyle[
	headfont=\bfseries\color{RawSienna},
	mdframed={style=mdredbox},
	headpunct={\\[3pt]},
	postheadspace={0pt},
]{thmredbox}

\declaretheorem[style=thmbluebox,name=مبرهنة,numberwithin=chapter]{theorem}
\declaretheorem[style=thmbluebox,name=تـوطـئـة,sibling=theorem]{lemma}
\declaretheorem[style=thmbluebox,name=خـاصـيـة,sibling=theorem]{proposition}
\declaretheorem[style=thmbluebox,name=نـتـيـجـة,sibling=theorem]{corollary}
\declaretheorem[style=thmbluebox,name=مبرهنة و تعريف]{theoremdef}
\declaretheorem[style=thmredbox,name=مثال,numberwithin=chapter]{example}

\mdfdefinestyle{mdgreenbox}{%
	skipabove=8pt,
	linewidth=2pt,
	rightline=true,
	leftline=false,
	topline=false,
	bottomline=false,
	linecolor=ForestGreen,
	backgroundcolor=ForestGreen!5,
}
\declaretheoremstyle[
	headfont=\bfseries\sffamily\color{ForestGreen!70!black},
	bodyfont=\normalfont,
	spaceabove=2pt,
	spacebelow=1pt,
	mdframed={style=mdgreenbox},
]{thmgreenbox}
\declaretheoremstyle[
	headfont=\bfseries\sffamily\color{ForestGreen!70!black},
	bodyfont=\normalfont,
	spaceabove=2pt,
	spacebelow=1pt,
	mdframed={style=mdgreenbox},
	headpunct={},
]{thmgreenbox*}

\mdfdefinestyle{mdblackbox}{%
	skipabove=8pt,
	linewidth=3pt,
	rightline=true,
	leftline=false,
	topline=false,
	bottomline=false,
	linecolor=black,
	backgroundcolor=RedViolet!5!gray!5,
}
\declaretheoremstyle[
	headfont=\bfseries,
	bodyfont=\normalfont\small,
	spaceabove=0pt,
	spacebelow=0pt,
	mdframed={style=mdblackbox}
]{thmblackbox}

\theoremstyle{theorem}
\declaretheorem[name=سؤال,style=thmblackbox,numberwithin=chapter]{ques}
\declaretheorem[name=تمرين,style=thmblackbox,sibling=ques]{exercise}
\declaretheorem[name=ملاحظة,numberwithin=chapter,style=thmgreenbox]{remark}
\declaretheorem[name=ملاحظة,numberwithin=chapter,style=thmgreenbox*]{remark*}
\declaretheorem[name=خطوة,style=thmgreenbox]{step} % only used in Lebesgue int
\declaretheorem[name=طريقة,style=thmgreenbox]{method}

\theoremstyle{definition}
\newtheorem{claim}[theorem]{Claim}

\mdfdefinestyle{mdorangebox}{%
	roundcorner = 10pt,
	linewidth=1pt,
	skipabove=12pt,
	innerbottommargin=9pt,
	skipbelow=2pt,
	nobreak=true,
	linecolor=black,
	backgroundcolor=YellowOrange!5,
}
\declaretheoremstyle[
	headfont=\sffamily\bfseries\color{black},
	mdframed={style=mdorangebox},
	headpunct={\\[3pt]},
	postheadspace={0pt}
]{thmorangebox}
\declaretheorem[name=تعريف,numberwithin=chapter,style=thmorangebox]{definition}

\newtheorem{fact}[theorem]{Fact}
\newtheorem{abuse}[theorem]{Abuse of Notation}

\newtheorem{problem}[exercise]{مسألـة}
\renewcommand{\theproblem}{\arabic{problem}}
\newtheorem{sproblem}[problem]{مسألـة}
\newtheorem{dproblem}[problem]{مسألـة}
\renewcommand{\thesproblem}{\theproblem$^{\star}$}
\renewcommand{\thedproblem}{\theproblem$^{\dagger}$}
\newcommand{\listhack}{$\empty$\vspace{-2em}}


% \mtcsetfont{parttoc}{chapter}{\sffamily\bfseries}
% \mtcsetfont{parttoc}{section}{\footnotesize\rmfamily\upshape\mdseries}
% \mtcsetfont{parttoc}{subsection}{\footnotesize\rmfamily\upshape\mdseries}
% %\mtcsetdepth{parttoc}{1}
% \setcounter{parttocdepth}{1}
% \renewcommand*{\partheadstartvskip}{\vspace*{20em}}
% \renewcommand*{\partheadendvskip}{}
% %\noptcrule
% \renewcommand\beforeparttoc{\noindent{\bfseries \Large Part \thepart: Contents}}
% %\hspace{\fill}\rule{0.95\linewidth}{2pt}\hspace{\fill}
% \doparttoc[n]

% %%fakesection Misc haxx
% \pdfstringdefDisableCommands{\def\Spec{\text{Spec }}}
